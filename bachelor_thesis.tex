% 卒業論文テンプレート

\documentclass[10pt]{jbook}

\usepackage{bachelor_thesis}

% A4 : width 21cm, height 29.7cm
\usepackage[a4paper,textheight=21.7cm,top=3cm,left=2cm,right=2cm,headheight=15pt]{geometry}

\年度{2021}
\学期{秋}
\題目{SMTソルバを用いた\\一次元NNA回路のコスト削減手法}
\指導教員{山下 茂}
\クラス{AA}
\コース{システムアーキテクト}
% \学科{情報システム}
% \回生{4}
\学籍番号{2600180159-6}
\氏名{清野恭平}


\begin{document}
% 表紙 %%%
\maketitle
%目次%
\tableofcontents
aa
% 第一章 %
\chapter{はじめに}

\begin{itemize}
  \item どのような分野の研究か,その背景について説明する.
     \begin{itemize}
       \item 量子コンピューターは量子力学の基本原則を用いることで,古典コンピューターが答えの導出に莫大な時間を要する問題を,線形時間で解くことができる.
       近年,クラウドで利用可能な量子コンピュータAPI,IBM Qシリーズの登場により,より研究が盛んに行われている. 
     \end{itemize}
  \item その分野の従来の研究状況について説明する.
     \begin{itemize}
      \item 量子回路を構成するためにはMCTゲートやToffoliゲートなど様々な2入力以上の量子ゲートが用いられる.
      しかし,量子コンピュータ状で実際に動作する量子ゲートはCNOTゲートと1量子bitゲートのみである.
      \item そのため、最終的にはCNOTゲートと1量子bitゲートに分解され,このゲート群はClifford+Tゲート群と呼ばれる.
      \item Clifford+Tゲート群はさらにNNA制約と呼ばれる制約を満たすように分解される.
      NNA制約とは,隣接する量子ビット間でしかCNOTゲートを作用できない制約である.
      \item 
    \end{itemize}
  \item そして,何が解決すべき問題(本論文で扱った問題)かを説明.
    \begin{itemize}
      \item Clifford+Tゲート群をNNA向け回路に分解すると,分解前と比べてゲート数が増加する.これをNNCと呼ぶ.
      \item NNCが増加するとエラー率が上昇するため,NNCは出来るだけ削減するのが望ましい.
      \item 既存手法としてSMTソルバを用いてNNA制約を満たす最小構成を求める手法がある.
      まずはClifford+Tゲート群を前から分割して,それぞれの入力と満たすべき出力を得る.
      分割した回路の入力は直前の分割した回路の出力になる.
      その後,それぞれの回路において入力とNNA制約の条件式をSMTソルバに与え,分解後回路の出力が,分解前の回路の出力と一致するような最小のCNOTゲート群を得る.
      それらを結合することにより,最終的にNNA制約を満たすゲート群を得る.
      \item 1量子ビットを作用させるまでは,量子ビットの状態は考慮しなくてもよい.1量子bitゲートを作用する際に,要求された量子状態を
    \end{itemize}
  \item どのようなアイデアで解決したか,キーアイデアを少しだけ披露
    \begin{itemize}
      \item 提案手法では,1量子ビットゲートが作用するまで,量子状態を考慮しない.これにより,分割した回路における要求出力の中にdon't careなbitが生じる.
    \end{itemize}
  \item どのような(実験)結果が得られたか、アピール(目次案の段階では希望的予測)
  \begin{itemize}
    \item 提案手法の評価のため,提案手法をpythonで実装した.
    \item 提案した手法をベンチマークに適応した.
    \item 既存の手法と比較し,NNCが30%削減できた(希望)
  \end{itemize}
 \end{itemize}




 


量子コンピューターにおいて2量子ゲートは物理的に隣接する量子ビットにしか適応できず,これを「Nearest Neighbor 制約」(NN 制約)と呼ぶ.
量子回路を構成する際は,まずはNN制約を無視し.可逆回路でで実現したい関数を表現する.
その後に,対象の量子アーキテクスチャで動作するように,NN制約を満たすようなゲート群に分解を行う.
しかし,大抵の場合で元の回路よりもゲート数が増大しコストが大きくなってしまう.コストが増えるとエラー率が上昇するため,より少ないゲート数でNNA量子回路を実現することが求められる.

%NN制約を満たすためSwAPゲートをもちいて,量子ビットの量子状態を入れ替える事で,コントロールビットとターゲットビットを隣接させ,演算を行う手法が知られる.%





NNA量子回路のコストを減らす手法として,SATソルバを用いたものが知られる.
SATソルバを用いることで,厳密にゲート数を最小にすることが可能である.
しかし,既存のSATソルバを用いた手法は単位ゲートのbit間の移動と,回路途中のdon’t careを考慮していない.

 
本論文では,単位ゲートの移動と,回路途中のdon’t careを考慮することにより,ゲート数を削減する手法を紹介する.
また,既存手法ではSATソルバを用いていたが,本論文ではSMTソルバを利用する.
量子回路では,単位ゲートを作用させるまで,量子bitの論理状態は~であるため,これらを考慮することでゲート数が減少することが考えられる.


 

実験の結果大幅にゲート数を削減することが出来た.


%第2章%
\chapter{基礎知識}
\section{量子回路}
\subsection{量子ゲート}
\subsubsection{Hゲート}
\subsubsection{CNOTゲート}
\subsection{NNA}
\subsection{garbage bit}
\section{充足可能性問題}
\subsection{SATソルバ}
\subsection{SMTソルバ}
%第3章%
\chapter{SMTソルバを用いた一次元NNA回路のコスト削減手法}
\section{要求出力生成}
\section{SMTソルバを用いた要求出力を満たすようなNNA回路の構成}
\section{単位ゲートの入れ替えとdon't careの考慮}
%第四章%
\chapter{実験}
\section{評価方法}
\section{実験結果と考察}
%第五章%
\chapter{おわりに}
\section{本研究のまとめ}
\section{今後の課題}



\end{document}
%章 テンプレート
/%    
\chapter{はじめに}
\section{節の名前}
\subsection{小節の名前}
 %/


